% preamble %
\documentclass[12pt]{article}
\usepackage{amsfonts}
\usepackage{fancyhdr}
\usepackage{comment}
\usepackage[a4paper, top=1cm, bottom=1.5cm, left=2cm, right=2cm]{geometry}
\usepackage{enumitem}
\usepackage{times}
\usepackage{changepage}
\usepackage{amssymb}
\usepackage{graphicx}
\usepackage{tabularx}
\usepackage{titlesec}

% settings %
\renewcommand{\contentsname}{Innholdsfortegnelse}
\setcounter{secnumdepth}{4}

% commands %
\titleformat{\paragraph}
{\normalfont\normalsize\bfseries}{\theparagraph}{1em}{}
\titlespacing*{\paragraph}
{0pt}{3.25ex plus 1ex minus .2ex}{1.5ex plus .2ex}

% document %
\begin{document}
\title{%
    Kravspesifikasjon\\
    \large Parkeringsplass reservasjons-app }
\author{Gruppe 12}
\date{}
\maketitle

\newpage

\tableofcontents

\newpage

% Hensikt %
\section{Probelm stilling}
% Dette er en ide for hva man kan bruke i denne problem stillingen, en av de "Type x" vil bli igjen eller slått sammen %

% Gamle %
    \subsection{Type 1}
    En oppstartsbedrift ønsker å utvikle en tjeneste basert på konseptet om delingsøkonomi. Siden parkeringsplasser er en ressurs som tilstrekkelig, valgte de det. Du har sikkert vært på utkikk etter en ledig parkeringsplass den dagen du skulle shoppe med noen venner, delta i et viktig møte eller rekke et tog, men samme hvor du ser står det en bil. Det kan være både frustrerende og leit, spesielt når tiden er avgjørende. Dette ønsker en oppstartsbedrift å løse ved å lage en tjeneste der enkeltpersoner og bedrifter kan leie ut parkeringsplassene sine til andre.

% Nye %
    \subsection{Type 2}
    Mange har sikkert slitet med å finne en parkeringsplass ett eller annet sted, om det skulle ha vert på et kjøpesenter, mitt i byen eller det å finne det best tilbudet som passer seg selv best. Det kan også hende at noen vil også bli en del av markede, i å eie sinn egen parkeringsplass som en enkeltperson eller som en bedrift, der det er enkelt og greit å sette opp parkeringsplass for leier ut til andre personer som trenger en kjapp og enkel parkeringsplass. 

% Scope %
\section{Scope}

% Definisjoner %
\section{Definisjoner}

\begin{center}
    \begin{tabular}{|p{4cm}|p{12cm}| } 
        \hline
        \bf Begrep & \bf Beskrivelse \\
        \hline
        Applikasjon &  Et programvare som benytter datamaskinens ressurser til en oppgave som brukeren ønsker utført \\
        \hline
        Bruker & Besøkende som samhandler med systemet \\
        \hline
        Brukerkonto & En som kan kun leie parkeringsplasser \\
        \hline
        Bedriftskonto & En som kan kun leie ut parkeringsplasser til andre \\
        \hline
        Feed & Opplisting av innlegg \\
        \hline
        ID & Identifikator \\
        \hline
        Tjenesten & Tjenesten denne dokumentasjonen omhandler \\
        \hline
        Systemet & Det underliggende systemet til tjenesten \\
        \hline
        Personas & Er et annet ord for rollefigur, enkeltmenneske, karakterpersonlighet\\
        \hline
    \end{tabular}
\end{center}

% Referanser %
\section{Referanser}

% Presonas %
\section{Personaer}

    \subsection{Persona type 1}
    \textbf{Alder:} 34 \\\textbf{Jobb/Rolle:} Økonom\\\textbf{Beskrivelse:}\\Jobber som økonom. Han skal grave i gården og lage ny parkeringsplass. Han vet ikke hvor lang tid det kommer til å ta, og han kommer ikke til å ha et sted å parkere i mellomtiden.\\\textbf{Motivasjon:}\\Han har ikke parkeringsplass hjemme, så han trenger en parkeringsplass i nærheten.\\\textbf{Behov:}\\Han må kunne forlenge antall dager han låner en parkeringsplass\\\textbf{Tekniske ferdigheter:} Gode

    \subsection{Persona type 2}
    \textbf{Alder:} 42\\\textbf{Jobb/Rolle:} Ledelse i bedrift\\\textbf{Beskrivelse:}\\Jobber i ledelsen i en bedrift. Det er ikke alltid han får en plass til å parkere i byen når han skal på jobb.\\\textbf{Motivasjon:}\\ Parkeringsplassene i byen er fullstappa.\\\textbf{Behov:}\\Han må kunne parkere nærmere arbeidsplassen og sikre seg parkeringplass ved et møte.\\\textbf{Tekniske ferdigheter:} Gode

    \subsection{Persona type 3}
    \textbf{Alder:} 22\\\textbf{Jobb/Rolle:} Uføre\\\textbf{Beskrivelse:}\\Karina er en rullestolbruker som trenger handicap-parkering. Hun vil helst parkere så nærme målet sitt som mulig, men vet ikke om det finnes en ledig handicap-parkeringsplass. Hun vil heller ikke risikere at stedet ikke tilbyr parkeringsplass for handicappede.\\\textbf{Motivasjon:}\\Hun trenger en oversikt over ledige handicap-parkeringsplasser.\\\textbf{Behov:}\\Hun må være sikker på at hun får en handicap-parkeringsplass.\\\textbf{Tekniske ferdigheter:} Gode

    \subsection{Persona type 4}
    \textbf{Alder:} 34 \\\textbf{Jobb/Rolle:} Selger\\\textbf{Beskrivelse:}\\Karin er dør-til-dør selger, hun kjører til nabolagene hun jobber i hver dag, men det er vanskelig å finne parkering, som gjør at hun taper tid og penger mens hun leter. Derfor åpner hun appen og booker parkeringsplasser for områdene hun skal jobbe den neste måneden. Nå føler Karin at hun har en mindre ting å tenke på.\\\textbf{Motivasjon:}\\Han har ikke parkeringsplass hjemme, så han trenger en parkeringsplass i nærheten\\\textbf{Mål:}\\Karin ønsker å eliminere en del av hverdagen som er unødvendig og som koster henne tid og penger. Karin forventer at hun skal finne parkering, og kunne booke det for måneden fremover, eller på dagen om nødvendig.\\\textbf{Reisen:}\\For å oppnå dette er Karin nødt til å finne utleiere som ønsker kort-tid/dagsutleie, så hun er nødt til å søke området hun skal til, og filtrere til dette alternativet. Deretter kan hun velge sted å parkere basert på beliggenhet og pris i forhold til området hun utvalgte seg.\\\textbf{Følelser:}\\Karin føler seg frustrert når hun kjører rundt nabolaget hun skal banke dører i, men aldri klarer å finne et sted det er lov å parkere. Karin er på sitt lykkeligste når hun åpner appen for å sjekke hvor hun har bestilt parkering, og vet at hun ikke trenger å tenke på å lete igje\\\textbf{Tekniske ferdigheter:} Gode

    \subsection{Persona type 5}
    \textbf{Alder:} 56\\\textbf{Jobb/Rolle:} Bonde\\\textbf{Beskrivelse:}\\Jørn bor i nærheten av et vakkert tjern som er populært blant unge å bade i, men siden det ikke er noen parkeringsplass der må folk parkere langs veien. Jørn eier en fin tomt i nærheten der folk kan parkere tryggere.\\\textbf{Motivasjon:}\\Han vil at tilby parkeringsplasser slik at ungdommene mer sannsynligvis kommer tilbake til tjernet og bader.\\\textbf{Behov:}\\En omtale system som kan bidra å spre det gode rykte videre. Han vil leie ut tomten for en inntekt ved siden av gårsdriften.\\\textbf{Tekniske ferdigheter:} Dårlig

% Funksjonelle krav %
\section{Funksjonelle krav}

    \subsection{Bruker klasse: Brukerkonto}

        \subsubsection{Innlogging}
            Bruken må oppgi brukernavn, passord, e-post og telefonnummer, hvis ikke de har allerede lagret en bruker fra før.

            \paragraph{Systemet verifiserer registreringen}
            Når bruker har registrert konto skal de bli tilsendt 4-sifret kode via e-post. Når koden er skrevet inn i tekst-feltet på nettsiden skal dataen lagres i en database.

            \paragraph{Derigere opprettet bruker}
            Etter opprettelse av konto skal bruker bli ført til innloggingssiden.

            \paragraph{Derigere innlogget bruker}
            Etter vellykket innlogging skal bruker bli ført til startsiden.

            \paragraph{Systemet skal verifisere innloggingen}
            Dersom innloggingsopplysningene er korrekte i forhold til opplysningene i brukeropprettingen, skal brukeren bli ført til forsiden. Dersom de er ukorrekte, skal brukeren få en feilmeldingen på hva som gikk galt.

            % Har ikke lyst til å ha med denne delen -Håkon %
            \paragraph{Bruker skal kunne logge inn}
            Bruker skal kunne velge å logge inn som en brukerkonto eller en bedriftskonto, og må deretter oppgi riktig innloggingsopplysninger.
        
        \subsubsection{Leie en parkeringsplass}

            \paragraph{Hvordan finne en parkeringsplass område}
            Hvis en bruker skal finne et sted for å parkere så kan de søke opp et område i applikasjonen og finne den plassen som passer best eller det å skrive inn naven på parkeringsplass området.

            \paragraph{Når og hvor kan en bruker leie}
            En bruker skal kunne leie hvor og når enn de vil, hvis ikke det er noen bestemte regler for parkeringsplass området.

            \paragraph{Hvordan leie en parkeringsplass}
            Som en brukerkonto skal man kunne leie parkeringsplasser, ved å velge en bil (som man har lagt til i systemet sitt selv eller registrere en bil), plass og hvor lang tid de skal stå der. Når brukeren har gjordt dette så skal de kunne betalle for parkeringsplassen.

            % Splitte opp i to underklasser??? -> tid og betaling%
            \paragraph{Tid og betaling}
            Hvis tiden på parkeringsplass holder på å gå ut, så vil brukeren som har leid parkeringsplassen få en melding på mobilen og/eller e-post om at den holde på å gå ut. Da kan de velge om å forlatte parkeringsplass sånn at enn annen kan ta over eller å fylle på med mer tide. Hvis brukeren kommer tilbake før tiden har gått ut skal brukeren få tilbake resten av pengene for den tiden de ikke sto der.

        \subsubsection{Komentere og rangering}

            \paragraph{Komentering}
            En bruker skal kunne skrive komentarer til en parkeringsplass de har vært på. 

            \paragraph{Rangering}
            En bruker skal kunne rangere en parkeringsplass fra 1 til 5 (der 1 er svært dårlig og 5 er kjempe bra), på en parkeringsplass de har vært på.

    \subsection{Bruker klasse: Bedriftskonto}

        \subsubsection{Innlogging}
            Bruker/Bedrift må oppgi unik brukernavn, passord, e-post, bedriftsnavn og organisasjonsnummer (evt. andre opplysinger om lovverk), hvis ikke de har lagt seg til i systemet.

            % Ta en titt%
            \paragraph{Systemet verifiserer registreringen}
            Når bruker har registrert konto skal de bli tilsendt 4-sifret kode via e-post. Når koden er skrevet inn i tekst-feltet på nettsiden skal dataen lagres i en database, det kan/vil følle opp mer verifisering når man skal lage en bedriftskonto.

            \paragraph{Derigere opprettet bruker}
            Etter opprettelse av konto skal bruker bli ført til innloggingssiden.

            \paragraph{Derigere innlogget bruker}
            Etter vellykket innlogging skal bruker bli ført til startsiden.

            \paragraph{Systemet skal verifisere innloggingen}
            Dersom innloggingsopplysningene er korrekte i forhold til opplysningene i brukeropprettingen, skal brukeren bli ført til forsiden. Dersom de er ukorrekte, skal brukeren få en feilmeldingen på hva som gikk galt.

        \subsubsection{Opprette parkeringsplasser}

            \paragraph{Bedriftskonto skal kunne registrere en eller flere parkeringsplasser}
            For å registrere et parkeringsplass område må de oppgi bredde og lengde i centimeter, antall parkeringsplasser, type parkeringsplass (om det er handikapparkering, eletriskparkering, osv.), etasjer, pris (dette skal kunne endres mauelt eller automastisk i etter kant), postaddresse, gateaddresse og gatenummer.

            \paragraph{Systemet skal knytte parkeringsplassen til en unik ID}
            Den nylig registrerte parkeringsplassen skal knyttes til en unik ID.

            \paragraph{Systemet skal knytte parkeringsplassen til en bruker}
            Den nylig registrerte parkeringsplassen skal knyttes til en bruker (eier av parkeringsplassen).

        \subsubsection{Betaling og regler}

            \paragraph{Pris}
            Eieren av en parkeringsplassen som kan justere prisen på parkeringsplass (i timen).

            \paragraph{Tid}
            Eieren av en parkeringsplassen skal kunne sette en max og min tid for hvor lenge en bil kan stå parkert på en parkeringsplass

            \paragraph{Regler}
            Eieren av en parkeringsplass skal kunne lage sinnge enge regler for hvordan parkering skal oppholde seg, om det er om tid og/eller kostnad endringer.
        
        \subsubsection{Hindre andre i å missbruker parkeringsplass applikasjonen}

            \paragraph{Banlyse}
            En bedriftskonto skal kunne banlyse en brukerkonto som missbruker applikasjonen. Dette kan/vil hinde en bruker som er banlyst fra å kunne leie fra dem igjen.

            \paragraph{Fjenre kommentarer}
            En bedriftskonto skal kunne fjenre innlegg, som har blit skrevet på sinn egen parkeringsplass.

    \subsection{Bruker klasse: Administrator}

        \subsubsection{Holde orden på brukerne}

            \paragraph{Banlyse}
            Skal kunne banlyse andre bruker (bedriftskontoer og brukerkontoer).

            \paragraph{Slette/fjenre gamle/inaktive brukere}
            Skal kunne slette/fjenre gamle/inaktive brukere som ikke er i bruk.

            \paragraph{Slette/fjenre kommentarer}
            Skal kunne slette/fjenre komentarer som er skrevet av andre brukere.

            \paragraph{Slette/fjerne rangering}
            Skal kunne slette/fjenre andres rangeringer til parkeringsplassen.
    
% Ikke funksjonelle krav %
\section{Ikke funksjonelle krav}

% Interface %
\section{Interface}

% Systemfunksjoner %
\section{Systemfunksjoner}


% Diagramer fra hver student %
\section{Diagramer}
% Vi må alle lage 2 forskjellige  diagrame hver -> dataflytdiagram, sekvensdiagram, tilstandsdiagram eller aktivitetsdiagram %

    \subsection{Charis}

    \subsection{Håkon}

    \subsection{Jonas}

    \subsection{Lars}

    \subsection{Mats}

\section{Estemering}

\section{Ta en titt på dette}
% Dette her trenger en plass %
    \subsection{Lagring av data}

    \subsubsection{Systemet skal lagre informasjon om parkeringsplasser}
    Systemet skal lagre all informasjon relatert til enhver parkeringsplass i en tabell i en database eller en lokal fil.

    \subsection{Lesing av data}

    \subsubsection{Systemet skal hente data om en spesifikk parkeringsplass via ID}
    Systemet skal kunne hente data om en spesifikk parkeringsplass via den spesifikke parkeringsplassen sin ID.

    \subsection{Feed}

    \subsubsection{Feeden skal vise brukerinnlegg}

    \subsubsection{Bruker skal kunne velge et filter på feeden}

    \subsection{Brukerinnlegg}

\end{document}