% preamble %
\documentclass[12pt]{article}
\usepackage{amsfonts}
\usepackage{fancyhdr}
\usepackage{comment}
\usepackage[a4paper, top=1cm, bottom=1.5cm, left=2cm, right=2cm]{geometry}
\usepackage{enumitem}
\usepackage{times}
\usepackage{changepage}
\usepackage{amssymb}
\usepackage{graphicx}
\usepackage{tabularx}
\usepackage{titlesec}
\usepackage{hyperref}

% settings %
\setcounter{secnumdepth}{2} % enumerate
\setcounter{tocdepth}{2}    % TOC entries
\renewcommand{\contentsname}{Innholdsfortegnelse}
\newcounter{esidcounter}
% settings for paragraph
% 
% \titleformat{\paragraph}
% {\normalfont\normalsize\bfseries}{\theparagraph}{1em}{}
% \titlespacing*{\paragraph}
% {0pt}{3.25ex plus 1ex minus .2ex}{1.5ex plus .2ex}

% commands %
\newcommand*{\EsID}{\stepcounter{esidcounter}\textbf{EsID: FK\arabic{esidcounter}\\}}

% document %
\begin{document}
\title{%
    Kravspesifikasjon\\
    \large Parkeringsplass reservasjons-app }
\author{Gruppe 12}
\date{}
\maketitle

\newpage

\tableofcontents

\newpage

% Hensikt %
\section{Problem stilling}

Ofte finner man ikke parkeringsplass mens man er i farta. Det er tett på senterene, gateparkering er privat eller ulovlig og man har ofte dårlig tid. Gjerne etter litt leting finner man en parkering 500 meter unna, og når man kommer dit er det selvfølgelig fullt. Hvordan kan man unngå å miste så mye tid, og i tillegg finne en parkeringsplass?

% Scope %
\section{Scope}

% Definisjoner %
\section{Definisjoner}

\begin{center}
    \begin{tabular}{|p{4cm}|p{12cm}| } 
        \hline
        \bf Begrep & \bf Beskrivelse \\
        \hline
        Applikasjon &  Et programvare som benytter datamaskinens ressurser til en oppgave som brukeren ønsker utført \\
        \hline
        Bruker & Besøkende som samhandler med systemet \\
        \hline
        Brukerkonto & En som kan kun leie parkeringsplasser \\
        \hline
        Bedriftskonto & En som kan kun leie ut parkeringsplasser til andre \\
        \hline
        EsID & Estemering identifikator \\
        \hline
        Feed & Opplisting av innlegg \\
        \hline
        ID & Identifikator \\
        \hline
        Tjenesten & Tjenesten denne dokumentasjonen omhandler \\
        \hline
        Systemet & Det underliggende systemet til tjenesten \\
        \hline
        Personas & Er et annet ord for rollefigur, enkeltmenneske, karakterpersonlighet\\
        \hline
        FK & Funksjonelt Krav (brukes til EsID)\\
        \hline
        Innlegg & Brukerinnlegg i systemet (post) som består av en eller flere parkeringsplasser\\
        \hline
    \end{tabular}
\end{center}

% Referanser %
\section{Referanser}

% Presonas %
\section{Personaer}

    \subsection{Persona type 1}
    \textbf{Alder:} 34 \\\textbf{Jobb/Rolle:} Økonom\\\textbf{Beskrivelse:}\\Jobber som økonom. Han skal grave i gården og lage ny parkeringsplass. Han vet ikke hvor lang tid det kommer til å ta, og han kommer ikke til å ha et sted å parkere i mellomtiden.\\\textbf{Motivasjon:}\\Han har ikke parkeringsplass hjemme, så han trenger en parkeringsplass i nærheten.\\\textbf{Behov:}\\Han må kunne forlenge antall dager han låner en parkeringsplass\\\textbf{Tekniske ferdigheter:} Gode

    \subsection{Persona type 2}
    \textbf{Alder:} 42\\\textbf{Jobb/Rolle:} Ledelse i bedrift\\\textbf{Beskrivelse:}\\Jobber i ledelsen i en bedrift. Det er ikke alltid han får en plass til å parkere i byen når han skal på jobb.\\\textbf{Motivasjon:}\\ Parkeringsplassene i byen er fullstappa.\\\textbf{Behov:}\\Han må kunne parkere nærmere arbeidsplassen og sikre seg parkeringplass ved et møte.\\\textbf{Tekniske ferdigheter:} Gode

    \subsection{Persona type 3}
    \textbf{Alder:} 22\\\textbf{Jobb/Rolle:} Uføre\\\textbf{Beskrivelse:}\\Karina er en rullestolbruker som trenger handicap-parkering. Hun vil helst parkere så nærme målet sitt som mulig, men vet ikke om det finnes en ledig handicap-parkeringsplass. Hun vil heller ikke risikere at stedet ikke tilbyr parkeringsplass for handicappede.\\\textbf{Motivasjon:}\\Hun trenger en oversikt over ledige handicap-parkeringsplasser.\\\textbf{Behov:}\\Hun må være sikker på at hun får en handicap-parkeringsplass.\\\textbf{Tekniske ferdigheter:} Gode

    \subsection{Persona type 4}
    \textbf{Alder:} 34 \\\textbf{Jobb/Rolle:} Selger\\\textbf{Beskrivelse:}\\Karin er dør-til-dør selger, hun kjører til nabolagene hun jobber i hver dag, men det er vanskelig å finne parkering, som gjør at hun taper tid og penger mens hun leter. Derfor åpner hun appen og booker parkeringsplasser for områdene hun skal jobbe den neste måneden. Nå føler Karin at hun har en mindre ting å tenke på.\\\textbf{Motivasjon:}\\Han har ikke parkeringsplass hjemme, så han trenger en parkeringsplass i nærheten\\\textbf{Mål:}\\Karin ønsker å eliminere en del av hverdagen som er unødvendig og som koster henne tid og penger. Karin forventer at hun skal finne parkering, og kunne booke det for måneden fremover, eller på dagen om nødvendig.\\\textbf{Reisen:}\\For å oppnå dette er Karin nødt til å finne utleiere som ønsker kort-tid/dagsutleie, så hun er nødt til å søke området hun skal til, og filtrere til dette alternativet. Deretter kan hun velge sted å parkere basert på beliggenhet og pris i forhold til området hun utvalgte seg.\\\textbf{Følelser:}\\Karin føler seg frustrert når hun kjører rundt nabolaget hun skal banke dører i, men aldri klarer å finne et sted det er lov å parkere. Karin er på sitt lykkeligste når hun åpner appen for å sjekke hvor hun har bestilt parkering, og vet at hun ikke trenger å tenke på å lete igje\\\textbf{Tekniske ferdigheter:} Gode

    \subsection{Persona type 5}
    \textbf{Alder:} 56\\\textbf{Jobb/Rolle:} Bonde\\\textbf{Beskrivelse:}\\Jørn bor i nærheten av et vakkert tjern som er populært blant unge å bade i, men siden det ikke er noen parkeringsplass der må folk parkere langs veien. Jørn eier en fin tomt i nærheten der folk kan parkere tryggere.\\\textbf{Motivasjon:}\\Han vil at tilby parkeringsplasser slik at ungdommene mer sannsynligvis kommer tilbake til tjernet og bader.\\\textbf{Behov:}\\En omtale system som kan bidra å spre det gode rykte videre. Han vil leie ut tomten for en inntekt ved siden av gårsdriften.\\\textbf{Tekniske ferdigheter:} Dårlig

% Funksjonelle krav %
\section{Funksjonelle krav}

    \subsection{Kontoopprettelse}
        
        \subsubsection{Bruker skal kunne opprette en brukerkonto}
        \EsID
        For å oppgi en brukerkonto, må bruker oppgi unikt brukernavn, passord, e-post og telefonnummer.

        \subsubsection{Bruker skal kunne opprette en bedriftskonto}
        \EsID
        For å opprette en bedriftskonto, må bruker må oppgi unikt brukernavn, passord, e-post, bedriftsnavn og organisasjonsnummer.

        \subsubsection{Systemet skal sende verifiseringskode til bruker}
        \EsID
        Etter registrering av konto skal bruker bli tilsendt en 4-sifret kode via e-post, samt midlertidig lagre denne koden i databasen. Koden skrives inn i et tekst-felt for verifisering.

        \subsubsection{Systemet skal verifisere registreringen}
        \EsID
        Den 4-sifra koden fra brukeren skal sjekkes opp den eksisterende koden i databasen, hvis den stemmer er kontoopprettelsen verifisert, og koden slettes fra databasen.

        \subsubsection{Systemet skal lagre dataen til den brukeren}
        \EsID
        Etter at brukeropprettelsen er verifisert, blir data fra opprettelsen lagret i databasen.

        \subsubsection{Derigere opprettet bruker}
        \EsID
        Etter opprettelse av konto skal bruker bli ført til startsiden.
    
    \subsection{Innlogging}

        \subsubsection{Bruker skal kunne logge inn som en brukerkonto eller bedriftskonto}
        \EsID
        Bruker skal kunne velge å logge inn som en brukerkonto eller bedriftskonto, og så fylle ut tekst-felter med data som stemmer i forhold til opprettelsen av kontoen.

        \subsubsection{Systemet skal verifisere innloggingen}
        \EsID
        Dersom innloggingsopplysningene er korrekte i forhold til opplysningene i brukeropprettingen, skal brukeren bli logget inn. Dersom de er ukorrekte, skal brukeren få en feilmeldingen på hva som gikk galt.

        \subsubsection{Derigere innlogget bruker}
        \EsID
        Etter vellykket innlogging skal bruker bli ført til startsiden.

    \subsection{Presentasjon av parkeringsplasser (Feed)}

        \subsubsection{Feeden skal vise en liste over relevante parkeringsplasser}
        \EsID
        Opplistingen skal bestå av innlegg av parkeringsplasser fra andre brukere.
        
        \subsubsection{Bruker skal kunne sette et filter på feeden}
        \EsID
        Opplistingen rangeres via et brukervalgt filter. Filteret bestemmer om opplastingen rangeres basert på nærmest avstand, lavest til høyest pris, høyest til lavest pris, parkeringsplasstype og om den leies ut av en brukerkonto eller bedriftskonto. Standardinnstillingen skal være nærmest avstand.

        \subsubsection{Bruker skal kunne søke etter en adresse}
        \EsID
        Feeden skal rangere parkeringsplassene basert på avstand i forhold til denne adressen.

        \subsubsection{Bruker skal kunne søke etter koordinater}
        \EsID
        Feeden skal rangere parkeringsplassene basert på avstand i forhold til koordinatene.

    \subsection{Leie parkeringsplass}

        \subsubsection{Bruker skal kunne leie parkeringsplass som en brukerkonto}
        \EsID
        Som en brukerkonto skal bruker kunne leie en spesifikk parkeringsplass eller flere parkeringsplasser.

        \subsubsection{Utløp av tid}
        \EsID 
        Når en reservert parkeri holder på å gå tom for tid, så skal brukeren som har leid parkeringsplassen få en varsel om at den holder på å gå ut.

    \subsection{Tilbakemelding på innlegg}

        \subsubsection{Bruker skal kunne gi tilbakemelding på et innlegg}
        \EsID
        Dersom brukeren har brukt en parkeringsplass, skal de kunne gi tilbakemelding på innlegget til den parkeringsplassen. Innlegget må minst bestå av en poengsum fra 1 til 5, der 5 er den høyeste poengsummen. Om ønskelig kan bruker også legge til en kommentar på maks 130 tegn. Tilbakemeldingen skal lagres i databasen og tilknyttes innlegget.

        \subsubsection{Tilbakemeldinger skal vises på hvert innlegg}
        \EsID
        Alle innlegg som tilhører et innlegg skal vises som en liste på innlegget, der nyeste tilbakemelding ligger øverst. Hver tilbakemelding skal vise poengsummen, kommentaren og brukernavn til de som lagde den. I tillegg skal en endelig rangering vises på innlegget som består av gjennomsnittet av poengsummer fra alle tilbakemeldinger knyttet til innlegget.

        \subsubsection{Utleier skal kunne få tilbakemeldinger via mail}
        Ved valg har utleier mulighet til å få alle tilbakemeldinger sendt på mail.

    \subsection{Opprette parkeringsplasser}

        \subsubsection{Bedriftskonto skal kunne registrere en eller flere parkeringsplasser}
        \EsID
        For å registrere et parkeringsplass område må de oppgi bredde og lengde i centimeter, antall parkeringsplasser, type parkeringsplass (om det er handikapparkering, eletriskparkering, osv.), etasjer, pris (dette skal kunne endres mauelt eller automastisk i etter kant), postaddresse, gateaddresse og gatenummer.

        \subsubsection{Systemet skal knytte parkeringsplassen til en unik ID}
        \EsID
        Den nylig registrerte parkeringsplassen skal knyttes til en unik ID.
        
        \subsubsection{Systemet skal lagre parkeringsplassen i en database}
        \EsID
        Den nylig registrerte parkeringsplassen lagres i en database, der den knyttes til en bruker (eier av parkeringsplassen).

    \subsection{Leie betingeleser}

        \subsubsection{Pris}
        \EsID
        Eieren av en parkeringsplassen skal kunne justere pris på sin(e) parkeringsplass(er) (i timen).

        \subsubsection{Tid}
        \EsID
        Eieren av en parkeringsplassen skal kunne sette en max og min tid for hvor lenge en bil kan stå parkert på en parkeringsplass

        \subsubsection{Regler}
        \EsID
        Eieren av en parkeringsplass skal kunne lage sine egne regler for hvordan parkering skal oppholde seg, om det er om tid og/eller kostnad endringer.
        
    \subsection{Betaling}

        \subsubsection{Betale for en parkeringsplass}
        \EsID
        Brukeren betaler for parkeringsplassen via valgt betalingsmetode.

        \subsubsection{Tredjepart}
        \EsID
        Når en bruker sender inn en forespørsel om å kunne leie en parkeringsplass, så må tredjepart verifisere kjøpet før en bruker kan leie.

    \subsection{Hindre andre i å missbruker parkeringsplass applikasjonen}

        \subsubsection{Banlyse}
        \EsID
        En bedriftskonto skal kunne banlyse en brukerkonto som missbruker applikasjonen. Dette kan/vil hinde en bruker som er banlyst fra å kunne leie fra dem igjen.

        \subsubsection{Fjerne kommentarer}
        \EsID
        En bedriftskonto skal kunne fjerne innlegg, som har blit skrevet på sin egen parkeringsplass.

    \subsection{Administrering}

        \subsubsection{Administrator skal kunne bannlyse bruker}
        \EsID
        Administrator skal ha full rett til å bannlyse en vilkårlig bruker. Den bannlyste brukeren skal få beskjed om bannlysingen via e-post og direkte i systemet. 

        \subsubsection{Administrator skal kunne slette bruker}
        \EsID
        Administrator skal ha full rett til å slette individuelle brukere slik at all tilhørende data blir også slettet fra databasen. Den slettede brukeren skal få beskjed om slettingen via e-post.

        \subsubsection{Administrator skal kunne slette tilbakemeldinger}
        \EsID
        Administrator skal ha full rett til å slette individuelle tilbakemeldinger på andre innlegg slik at de fjernes både fra grensesnittet og databasen.

        \subsubsection{Administrator skal kunne slette innlegg}
        \EsID
        Administrator skal ha full rett til å slette individuelle innlegg. Bruker som lagde innlegget skal få beskjed om slettingen via e-post og direkte i systemet.

        \subsubsection{Administrator skal kunne slette parkeringsplasser}
        \EsID
        Administrator skal ha full rett til å slette individuelle parkeringsplasser. Bruker som opprettet parkeringsplassen skal få beskjed om slettingen via e-post og direkte i systemet.

        \subsubsection{Administrator skal kunne slette reservasjoner}
        \EsID
        Bla bla bla skirv etterpå

        \subsubsection{Bruker skal kunne rapportere andre brukere sine innlegg}
        \EsID
        Brukere har frihet til å rapportere andre brukere sine innlegg. Rapporteringen skal bestå av en begrunnelse. Brukeren som har lagd innlegget skal få varsel via e-post om rapporteringen.

        \subsubsection{Bruker skal kunne rapportere andre brukere}
        \EsID
        Brukere har frihet til å rapportere andre brukere. Rapporteringen skal bestå av en begrunnelse. Brukeren som har lagd innlegget skal få varsel via e-post om rapporteringen.
    
    \subsection{Lagring av data}

        \subsubsection{Systemet skal lagre informasjon om parkeringsplasser}
        \EsID
        Systemet skal lagre all informasjon relatert til enhver parkeringsplass i en tabell i en database eller en lokal fil.

    \subsection{Lesing av data}

        \subsubsection{Systemet skal hente data om en spesifikk parkeringsplass via ID}
        \EsID
        Systemet skal kunne hente data om en spesifikk parkeringsplass via den spesifikke parkeringsplassen sin ID.

           
\section{Andre funksjoner}
    % Ikke funksjonelle krav %
    \subsection{Ikke funksjonelle krav}

        \subsubsection{Tilgjengelighet}
        All brukere skal ha tilgang til applikasjonen til enhver tid.
        
        \subsubsection{Sikkerhetskopi(Backup)}


        \subsubsection{Dependency på tredjepart}


        \subsubsection{Legal and licensing issues}


        \subsubsection{Nettverkstopologi}


        \subsubsection{Preformance / response time}


        \subsubsection{Personvern}


        \subsubsection{Kvalitet}


        \subsubsection{Sikkerhet}


        \subsubsection{Supportability}


        \subsubsection{Testability}
        

    % Interface %
    \subsection{Interface}

    % Systemfunksjoner %
    \subsection{Systemfunksjoner}

        \subsubsection{Fremtidige funksjoner}

            \subsubsection{Bruker kan ha flere biler}
            Når en kunde skal resevere en parkeringsplass, så kan det hende at de har tilgang på flere biler og de skal de bilen være lagret etter bruk i brukerns database.

            \subsubsection{Lagre/Favorit}
            En kunde kan bruke den samme parkeringsplassen flere ganger og vil sikkert ha en kjapp måtte å koble seg opp mot den bestemte parkeringsplassen.

% Diagramer fra hver student %
\section{Diagrammer}
% Vi må alle lage 2 forskjellige  diagrame hver -> dataflytdiagram, sekvensdiagram, tilstandsdiagram eller aktivitetsdiagram %

    \subsection{Charis}

    \newpage
    \subsection{Håkon}

    \includegraphics[scale=0.45]{BilderTilLatex/SE2020-Gruppe12-Sekvensdiagram-Håkon.png}


    \subsection{Jonas}

    \subsection{Lars}

    \subsection{Mats}

\section{Estemering}

    \subsection{Av funksjonelle krav for en ...}

        \subsubsection{Brukerkonto}
        \begin{tabular}{|p{2cm}|p{4cm}|p{4cm}|p{4cm}| } 
            \hline
            \bf EsID & \bf Vikitighet & \bf Vanskelighetsgrad & \bf Utviklingsstørrelse\\
            \hline
            6.1.1.1 & NULL & NULL & NULL\\
            \hline
            6.1.1.2 & NULL & NULL & NULL\\
            \hline
            6.1.1.3 & NULL & NULL & NULL\\
            \hline
            6.1.1.4 & NULL & NULL & NULL\\
            \hline
            6.1.1.5 & NULL & NULL & NULL\\
            \hline
            6.1.1.6 & NULL & NULL & NULL\\
            \hline
            6.1.2.1 & NULL & NULL & NULL\\
            \hline
            6.1.2.2 & NULL & NULL & NULL\\
            \hline
            6.1.2.3 & NULL & NULL & NULL\\
            \hline
            6.1.2.4 & NULL & NULL & NULL\\
            \hline
            6.1.2.5 & NULL & NULL & NULL\\
            \hline
            6.1.3.1 & NULL & NULL & NULL\\
            \hline
            6.1.3.2 & NULL & NULL & NULL\\
            \hline
        \end{tabular}
    
        \subsubsection{Bedriftkonto}
        \begin{tabular}{|p{2cm}|p{4cm}|p{4cm}|p{4cm}| } 
            \hline
            \bf EsID & \bf Vikitighet & \bf Vanskelighetsgrad & \bf Utviklingsstørrelse\\
            \hline
            6.2.1.1 & NULL & NULL & NULL\\
            \hline
            6.2.1.2 & NULL & NULL & NULL\\
            \hline
            6.2.1.3 & NULL & NULL & NULL\\
            \hline
            6.2.1.4 & NULL & NULL & NULL\\
            \hline
            6.2.1.5 & NULL & NULL & NULL\\
            \hline
            6.2.2.1 & NULL & NULL & NULL\\
            \hline
            6.2.2.2 & NULL & NULL & NULL\\
            \hline
            6.2.2.3 & NULL & NULL & NULL\\
            \hline
            6.2.3.1 & NULL & NULL & NULL\\
            \hline
            6.2.3.2 & NULL & NULL & NULL\\
            \hline
            6.2.3.3 & NULL & NULL & NULL\\
            \hline
        \end{tabular}

        \subsubsection{Administrator}
        \begin{tabular}{|p{2cm}|p{4cm}|p{4cm}|p{4cm}| } 
            \hline
            \bf EsID & \bf Vikitighet & \bf Vanskelighetsgrad & \bf Utviklingsstørrelse\\
            \hline
            6.3.1.1 & NULL & NULL & NULL\\
            \hline
            6.3.1.2 & NULL & NULL & NULL\\
            \hline
            6.3.1.3 & NULL & NULL & NULL\\
            \hline
        \end{tabular}
\section{Ta en titt på dette}

\end{document}