% preamble %
\documentclass[12pt]{article}
\usepackage{amsfonts}
\usepackage{fancyhdr}
\usepackage{comment}
\usepackage[a4paper, top=1cm, bottom=1.5cm, left=2cm, right=2cm]{geometry}
\usepackage{enumitem}
\usepackage{times}
\usepackage{changepage}
\usepackage{amssymb}
\usepackage{graphicx}
\usepackage{tabularx}

% settings %
\renewcommand{\contentsname}{Innholdsfortegnelse}

% commands %


% document %
\begin{document}
\title{%
    Kravspesifikasjon\\
    \large Parkeringsplass reservasjons-app }
\author{Gruppe 12}
\date{}
\maketitle

\newpage

\tableofcontents

\newpage

\section{Hensikt}
En oppstartsbedrift ønsker å utvikle en tjeneste basert på konseptet om delingsøkonomi. Siden parkeringsplasser er en ressurs som tilstrekkelig, valgte de det. Du har sikkert vært på utkikk etter en ledig parkeringsplass den dagen du skulle shoppe med noen venner, delta i et viktig møte eller rekke et tog, men samme hvor du ser står det en bil. Det kan være både frustrerende og leit, spesielt når tiden er avgjørende. Dette ønsker en oppstartsbedrift å løse ved å lage en tjeneste der enkeltpersoner og bedrifter kan leie ut parkeringsplassene sine til andre.

\section{Scope}

\section{Definisjoner}

\begin{center}
    \begin{tabular}{|p{4cm}|p{12cm}| } 
        \hline
        \bf Begrep & \bf Beskrivelse \\
        \hline
        Bruker & Besøkende som samhandler med systemet \\
        \hline
        Tjenesten & Tjenesten denne dokumentasjonen omhandler \\
        \hline
        Systemet & Det underliggende systemet til tjenesten \\
        \hline
        ID & Identifikator \\
        \hline
        Feed & Opplisting av innlegg \\
        \hline
    \end{tabular}
\end{center}

\section{Funksjonelle krav}

\subsection{Innlogging}

\subsubsection{Bruker skal kunne opprette en brukerkonto}
Bruker må oppgi brukernavn, passord og e-post som ikke allerede er lagret i systemet.

\subsubsection{Bruker skal kunne opprette en bedriftskonto}
Bruker må oppgi unike brukernavn, passord, e-post, bedriftsnavn og organisasjonsnummer som ikke allerede er lagret i systemet.

\subsubsection{Systemet verifiserer registreringen}
Når bruker har registrert konto skal de bli tilsendt 4-sifret kode via e-post Når koden er skrevet inn i tekst-feltet på nettsiden skal dataen lagres i en database.

\subsubsection{Derigere opprettet bruker}
Etter opprettelse av konto skal bruker bli ført til innloggingssiden.

\subsubsection{Bruker skal kunne logge inn}
Bruker skal kunne velge å logge inn som en brukerkonto eller en bedriftskonto, og må deretter oppgi riktig innloggingsopplysninger.

\subsubsection{Derigere innlogget bruker}
Etter vellykket innlogging skal bruker bli ført til startsiden.

\subsubsection{Systemet skal verifisere innloggingen}
Dersom innloggingsopplysningene er korrekte i forhold til opplysningene i brukeropprettingen, skal brukeren bli ført til forsiden. Dersom de er ukorrekte, skal brukeren få en feilmeldingen på hva som gikk galt.

\subsection{Opprette parkeringsplasser}

\subsubsection{Bruker skal kunne registrere en eller flere parkeringsplasser}
Bruker må oppgi minst bredde og lengde i centimeter, postaddresse, gateaddresse og gatenummer.

\subsubsection{Systemet skal knytte parkeringsplassen til en unik ID}
Den nylig registrerte parkeringsplassen skal knyttes til en unik ID.

\subsubsection{Systemet skal knytte parkeringsplassen til en bruker}
Den nylig registrerte parkeringsplassen skal knyttes til en bruker (eier av parkeringsplassen).

\subsection{Lagring av data}

\subsubsection{Systemet skal lagre informasjon om parkeringsplasser}
Systemet skal lagre all informasjon relatert til enhver parkeringsplass i en tabell i en database eller en lokal fil.

\subsection{Lesing av data}

\subsubsection{Systemet skal hente data om en spesifikk parkeringsplass via ID}
Systemet skal kunne hente data om en spesifikk parkeringsplass via den spesifikke parkeringsplassen sin ID.

\subsection{Feed}

\subsubsection{Feeden skal vise brukerinnlegg}

\subsubsection{Bruker skal kunne velge et filter på feeden}

\subsection{Brukerinnlegg}




\end{document}